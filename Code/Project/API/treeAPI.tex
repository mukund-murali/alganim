\startluacode
--Default Values

treedx=6 			-- x-change
treedy=5.0			-- y-change
treeScale=7			-- scale factor
treeRadius=1			-- node radius
treeTextSize=0.8		-- label size for node
treeHeight = 10			-- bounding box vertical length
treeHeightdraw=10		-- 
treeThreshold = 6		-- value after which nodes are displayed as dots
treeMultiplicationFactor = 2	-- Factor for spacing nodes
treeLabelColor = "black"	-- color of node label
treeNodeColor = color4		-- color of node


--Methods

--[[
displays the algorithm with the current line highlighted

@param list of numbers, the lines that have to be highlighted
--]]
function showTreePseudoCode(...)				
	context("boxjoin(a.sw=b.nw;);")
	for i=1,#treeProgram do
		context("boxit.l"..i.."(btex "..treeProgram[i].."  etex)((1,0.5,0.5),0.8);l"..i..".dy=4;")
		if i == 1 then
			context("l1.w = (-20bp,"..(#treeProgram*20).."bp);")
		end
		context("drawleftunboxed(l"..i..");")
	end
	argpos = 1
	while arg[argpos] do  
		context("fillboxed(l"..arg[argpos]..");")
		argpos = argpos + 1
	end 
end


--[[
Sets the multiplication factor as given

@param Factor the spacing factor to be set for spacing nodes
--]]
function setTreeMultiplicationFactor(Factor)
	treeMultiplicationFactor = Factor
end

--[[
Sets the tree scale

@param Scale the scale factor to be used for drawing the tree
--]]
function setTreeScale(Scale)
	treeScale = Scale
	setScale(treeScale)
end

--[[
Set the node radius of nodes in the tree

@param Radius the radius to be set for the nodes
--]]
function setTreeRadius(Radius)
	treeRadius = Radius
end

--[[
Fixes the height of bounding box

@param Height height that has to be set for the bounding box
--]]
function fixTreeHeight(Height)
	if Height then
		totTreeDepth=Height
		treeHeight=Height
		if treeHeight > treeThreshold then
			treeHeightdraw=(Height+1)*8
			if treeHeight>=7 and treeHeight<12 then
				setTreeScale(2.5)
				treeMultiplicationFactor=9
			elseif treeHeight>=12 then
				setTreeScale(1.5)
				treeMultiplicationFactor=13
			end
		else
			treeHeightdraw = (Height+1)*30
		end
	end
end

--[[
loads the default values set for various tree attributes
--]]
function loadTreeDefaults()
	treedx=6
	treedy=5.0
	treeScale=7
	treeRadius=1
	treeTextSize=0.8
	treeHeight = 10
	treeHeightdraw=10
	treeThreshold = 6
	treeMultiplicationFactor = 2
	scale = treeScale
	loadTreeColorDefaults()
end

--[[
loads the default color set for the tree
--]]
function loadTreeColorDefaults()
	treeLabelColor="black"
	treeNodeColor=color6
end

--[[
draw a single node of the tree


@param x x co-ordinate of the center point
@param y y co-ordinate of the center point
@param num the label that has to be written inside the circle
@param nodeColor the fill color for the node(optional)
@param labelColor the color the label(optional)
@param radius the radius of the circle(optional)
@param textSize size of the label(optional) in pp
--]]
function drawTreeNode(x,y,num,nodeColor,labelColor,radius,textSize)
	if radius then
		treeRadius = radius
	end
	if textSize then
		treeTextSize = textSize
	end
	if labelColor then
		treeLabelColor = labelColor
	end
	if nodeColor then
		treeNodeColor = nodeColor
	end
	print(treeNodeColor)
	if treeHeight <= treeThreshold then
		drawNode(x,y,num,treeNodeColor,treeLabelColor,treeRadius,treeTextSize)
	else
		context("draw ("..(x*scale)..","..(y*scale)..") withpen pencircle scaled 3pt withcolor "..treeNodeColor..";")
	end
	loadTreeColorDefaults()
end	

--[[
draws the tree

@param node the root
@param depth the depth of the tree
@param first 1 or 0 - specifies whether the vertical line has to be drawn to maintain the height of the tree
--]]
function drawTree(node, depth)
	if not node.parent then 
		drawHeight(treeHeightdraw)
	end
	if node.parent then
		treeSpaceOutFactor = 1	
		--print("depth = "..totTreeDepth)
		if totTreeDepth == 2 then 
			treeSpaceOutFactor = 58.5			
		end
		if totTreeDepth == 3 then 
			treeSpaceOutFactor = 19.5			
		end
		if totTreeDepth == 4 then 
			treeSpaceOutFactor = 8.35			
		end
		if totTreeDepth == 5 then 
			treeSpaceOutFactor = 3.875			
		end
		if treeHeight>=treeThreshold then
			treedx=((depth) * treeMultiplicationFactor )/(math.pow(2,(node.level-1)))
		else
			print("tree space out factor "..treeSpaceOutFactor)
			treedx = treeSpaceOutFactor*((math.pow(2, (depth-node.level)))/2)
		end
		if node.parent.left == node then		
			node.X = node.parent.X-treedx
			node.Y = node.parent.Y-treedy
			if treeThreshold > treeHeight then
				drawLine(node.X,node.Y,node.parent.X-(treeRadius/2),node.parent.Y-(treeRadius/2))
			else
				drawLine(node.X,node.Y,node.parent.X,node.parent.Y)
			end
		else
			node.X = node.parent.X+treedx
			node.Y = node.parent.Y-treedy
			if treeThreshold >= treeHeight then
				drawLine(node.X,node.Y,node.parent.X+(treeRadius/2),node.parent.Y-(treeRadius/2))
			else
				drawLine(node.X,node.Y,node.parent.X,node.parent.Y)
			end
		end
	end
	drawTreeNode(node.X,node.Y,node.key)

	if node.left.key then
		drawTree(node.left, depth)
	end
	if node.right.key then
		drawTree(node.right, depth)
	end
end

levelElem = 1
prevDepth = 1

--[[

--]]
function drawTreeWithArray(node, depth, first)
	print("current depth : "..depth)
	if first then 
		drawHeight(treeHeightdraw)
		levelElem = 1
	end

	levelElem = levelElem +1

	if node.parent then
		treeSpaceOutFactor = 1	
		print("depth = "..totTreeDepth)
		if totTreeDepth == 2 then 
			treeSpaceOutFactor = 58.5			
		end
		if totTreeDepth == 3 then 
			treeSpaceOutFactor = 19.5			
		end
		if totTreeDepth == 4 then 
			treeSpaceOutFactor = 8.35			
		end
		if totTreeDepth == 5 then 
			treeSpaceOutFactor = 3.875			
		end

		if treeHeight>treeThreshold then
			treedx=((depth) * 7.5 )/(math.pow(2,(node.level-1)))
		else
			print("depth "..treeSpaceOutFactor)
			treedx = treeSpaceOutFactor*((math.pow(2, (depth-node.level)))/2)
		end

		if node.parent.left == node then		
			node.X = node.parent.X-treedx
			node.Y = node.parent.Y-treedy
			if treeThreshold > treeHeight then
				drawLine(node.X,node.Y,node.parent.X-(treeRadius/2),node.parent.Y-(treeRadius/2))
			else
				drawLine(node.X,node.Y,node.parent.X,node.parent.Y)
			end
		else
			node.X = node.parent.X+treedx
			node.Y = node.parent.Y-treedy
			if treeThreshold > treeHeight then
				drawLine(node.X,node.Y,node.parent.X+(treeRadius/2),node.parent.Y-(treeRadius/2))
			else
				drawLine(node.X,node.Y,node.parent.X,node.parent.Y)
			end
		end
	end

	k={1,2,3,44,675,6}
	color={"white","white","white","white","white","white"}
	print("nodeX:"..node.X.."  nodeY : "..node.Y)	
	drawArray(k,color,{},{},{},{},treeScale*node.X,treeScale*node.Y,""..node.level..levelElem)

	if node.left.key then
		drawTreeWithArray(node.left, depth)
	end
	if node.right.key then
		drawTreeWithArray(node.right, depth)
	end
end

\stopluacode
