\input ../API/baseAPI
\input ../Algorithms/arrayAlgos
\input ../Algorithms/treeAlgos
\input ../Algorithms/listAlgos
\input ../Algorithms/graphAlgos

\usemodule[simpleslides]
          [style=Sunrise]

\setupTitle
  [ title={Algorithm Animation in PDF Documents},
   author={Manthosh Kumar T, Mukund M, Prakash E \crlf \crlf Supervisor \crlf R S Milton \crlf Department of CSE},
     date={3rd May 2013},
	titlealign=center,
	authoralign=center,
	datealign=center,
	color=(0.8,0.8,0),
  ]

\starttext
\placeTitle
		
\SlideTitle {Abstract}
\startitemize
\item Incorporate animated diagrams in PDF
    documents for educational purpose.
  \item  \ConTeXt \ (typesetting macro based on \TeX ) to
    generate the PDF documents.  
  \item Animation is a sequence of
    layers with capabilities to play, pause, resume and rewind.  
  \item A module for animation in \ConTeXt \ with JavaScript.
  \item Animate algorithms for well-known data structures. Automatic layout of diagrams for arrays, lists, trees, and graphs. 
\stopitemize


\SlideTitle{Design and Implementation}
\crlf
\midaligned
{\externalfigure[IA.png][height=9cm]}


\SlideTitle{Major Work}
\startitemize
\item Standardize the Low-Level API and demonstrate its use with examples to create the High-Level API.
\item Adapt the base JavaScript file of \ConTeXt \ which is used by the t-animation module to our animation needs.
\item Create an interface to display the algorithm with its statements highlighted in sync with the animation of the data structure diagrams.
\stopitemize

\SlideTitle{User Classification }
\crlf
\crlf
\crlf
\midaligned
{\externalfigure[UC.png][width=18cm]}

\SlideTitle{Layout API}
\startitemize
\item Arrays
\item Lists
\item Trees
\item Graphs
\stopitemize

\SlideTitle{Layout API --- Arrays}
		\startitemize
		\item Boxes joined together create an array structure.
		\item To join the boxes, we use the following command
		\definetyping[C][tab=2]
		\startC
			box.join(a.ne=b.nw;a.se=b.sw;)
		\stopC
		\stopitemize

		\crlf Example: \crlf 
		
		Sorting of an Array - Selection Sort
		\definetyping[C][tab=3]
		\startC
		selectionSort({51,31,4,224,23})
		\stopC

\SlideTitle{Arrays --- Examples}
	\midaligned
	{\startluacode
		selectionSort({51,31,4,224})
	\stopluacode}

\SlideTitle{Layout API --- Lists}
		\startitemize
			\item Two ways of representing Lists
				\startitemize
					\item Abstract or Circular
					\item Detailed or Rectangular
				\stopitemize
			\item Nodes are connected with arrow heads.
			\item Markers to show currently accessed node.
		\stopitemize
		\definetyping[C][tab=1]
		\startC
			deleteListNodes({4,18},{1,2,3,4,5,6,54,3,158,23,54,76,18})
		\stopC
		\startluacode
			deleteListNodes({4,18},{1,2,3,4,5,6,54,3 ,158,23,54,76,18})
		\stopluacode

\SlideTitle{Layout API --- Trees}
		\startitemize
			\item Binary Trees.
			\item Node at same depth are spaced out evenly in X axis.
			\item To accommodate more nodes in the tree diagram, as depth increases, spacing is reduced proportionately.
		\stopitemize
		\crlf Example : \crlf
		Insertion in Binary Search Tree
		\definetyping[C][tab=2]
		\startC
		bstCreate({50,25,75,12,37,63,87,6,18,31,43,57,69,81,93})
		\stopC

\SlideTitle{Trees - Example}
		\crlf
		\crlf
		\crlf
		\midaligned
		{\startluacode
			bstCreate({50,25,75,12,37,63,87,6,18,31,43,57,69,81,93})
		\stopluacode}

\SlideTitle{Layout API --- Graphs}
		\startitemize
			\item Force-Directed Algorithm – application of force to nodes and edges.
			\item Aesthetically pleasing - Less Crossings.
			\item Edges of almost equal length.
		\stopitemize
		\crlf Example : \crlf
		Applying Dijkstra's algorithm
		\definetyping[C][tab=2]
		\startC
		dijsktra(adjacencyMatrix,startVertex,endVertrex)
		\stopC

\SlideTitle{Graphs - Example}
		\crlf
		\crlf
		\midaligned
		{\startluacode
		cost={
--	 1,2,3,4,5,6,7,8,9,0,1,2,3,4,5,6,7,8,9,0
	{0,1,0,0,1,1,0,0,0,0,0,0,0,0,0,0,0,0,0,0},
	{1,0,1,0,0,0,1,0,0,0,0,0,0,0,0,0,0,0,0,0},
	{0,1,0,1,0,0,0,1,0,0,0,0,0,0,0,0,0,0,0,0},
	{0,0,1,0,1,0,0,0,1,0,0,0,0,0,0,0,0,0,0,0},
	{1,0,0,1,0,0,0,0,0,1,0,0,0,0,0,0,0,0,0,0},
	{1,0,0,0,0,0,0,0,0,0,0,0,1,1,0,0,0,0,0,0},
	{0,1,0,0,0,0,0,0,0,0,0,0,0,1,1,0,0,0,0,0},
	{0,0,1,0,0,0,0,0,0,0,1,0,0,0,1,0,0,0,0,0},
	{0,0,0,1,0,0,0,0,0,0,1,1,0,0,0,0,0,0,0,0},
	{0,0,0,0,1,0,0,0,0,0,0,1,1,0,0,0,0,0,0,0},
	{0,0,0,0,0,0,0,1,1,0,0,0,0,0,0,0,0,0,1,0},
	{0,0,0,0,0,0,0,0,1,1,0,0,0,0,0,0,0,1,0,0},
	{0,0,0,0,0,1,0,0,0,1,0,0,0,0,0,0,1,0,0,0},
	{0,0,0,0,0,1,1,0,0,0,0,0,0,0,0,1,0,0,0,0},
	{0,0,0,0,0,0,1,1,0,0,0,0,0,0,0,0,0,0,0,1},
	{0,0,0,0,0,0,0,0,0,0,0,0,0,1,0,0,1,0,0,1},
	{0,0,0,0,0,0,0,0,0,0,0,0,1,0,0,1,0,1,0,0},
	{0,0,0,0,0,0,0,0,0,0,0,1,0,0,0,0,1,0,1,0},
	{0,0,0,0,0,0,0,0,0,0,1,0,0,0,0,0,0,1,0,1},
	{0,0,0,0,0,0,0,0,0,0,0,0,0,0,1,1,0,0,1,0},
}
dijsktra(cost,1,3)
		\stopluacode}

\SlideTitle{Fitting Diagrams in a Page}
		\startitemize
			\item Diagrams have to be accommodated within the PDF page regardless of the number of nodes in a diagram.
			\item Scaling the nodes as the no. of nodes increases is useful.
			\startformula Scaling\ Factor \ \alpha \ \frac{1}{No.\ of Nodes} \stopformula
			\item But decreasing the scaling factor might alone is not enough to get the diagram inside the page. 
		\stopitemize

\SlideTitle{Fitting Diagrams in  a Page}
		\startitemize
			\item When there is a large number of nodes, the readability of the diagram decreases.
			\item When the number of nodes exceeds a threshold, represent them by dots. The threshold depends on the data structure. 
		\stopitemize
		\midaligned
		{\starttable[|l|l|l|]
			\HL
			\VL \bf{Data Structure} \VL \use{2} \bf{Threshold(in no. of nodes)} \VL \SR
			\HL
			\VL Array \VL \use{2} 80 \VL \AR
			\HL
			\VL \Lower(.5\lineheight){List} \VL Abstract \VL Detailed \VL \AR
			\DC \DL[2] \DR
			\VL \VL 80 \VL 60 \VL \AR
			\HL
			\VL Tree \VL \use{2} 6 (Depth of Tree) \VL \AR
			\HL
			\VL Graph \VL \use{2} 50 \VL \AR
			\HL
		\stoptable}

\SlideTitle{Animation API}
		\startitemize
			\item \ConTeXt \ uses fieldstack mechanism to create overlapping layers
(frames) written in JavaScript. This is crude and cannot be directly used
to create animations. A third-party module, t-animation, makes it usable
to create animations; yet, it has only minimal features.
			\item Added three priority levels for frames: 
				\crlf (i) Sequential Frames (ii) Normal frames, and (iii) Breakpoints.
			\item This leads to adding the following features
				\startitemize
					\item Automatically pause at breakpoints
					\item Automatically skip sequential frames
				\stopitemize
		\stopitemize

\SlideTitle{Animation API}
		\startitemize
			\item The module supports options to play, pause, go to - next frame,
previous frame, first frame, and last frame.
			\item We have added the following options 
				\startitemize
					\item Increase/Decrease animation speed
					\item Toggle automatic pasue at breakpoints
					\item Toggle skipping of sequential frames
					\item Go to --- next breakpoint, previous breakpoint
				\stopitemize
		\stopitemize

\SlideTitle{Performance Analysis}
	\crlf
	\crlf
	\midaligned
		{\starttable[|l|l|]
			\HL
			\VL \bf{No. of Frames} \VL \bf{Time to Compile} \VL \SR
			\HL
			\VL 53 \VL 0:04 \VL \AR
			\HL
			\VL 88 \VL 0:10 \VL \AR
			\HL
			\VL 392 \VL 1:05 \VL \AR
			\HL
			\VL 886 \VL 3:53 \VL \AR
			\HL
			\VL 1279 \VL 8:29 \VL \AR
			\HL
			\VL 1663 \VL 11:12 \VL \AR
			\HL
		\stoptable}

\SlideTitle{Conclusion}
\startitemize
	\item Efficient Low-levl API for the creation of data structure diagrams and animation. 
	\item Examples for creating algorithm animations using the Low-level API.
	\item Pseudo-Code is displayed alongside the animation, making it easy to follow.
\stopitemize

\SlideTitle{References}
\startitemize
	\item Personal correspondence with Wolfgang Schuster, the author of t-animation module of ConTEXt.
			\crlf E-mail: wolfgang.schuster@gmail.com
	\item Yifan Hu. Efficient and High-Quality Force-Directed Graph Drawing, Wolfram Research Inc., USA.
	\item John D. Hobby. Drawing Boxes with MetaPost. [Online]. \crlf Available: http://www.tug.org/docs/metapost/mpboxes.pdf
	\item Dennis Hotson. Springy.js – A force directed graph layout algorithm in JavaScript. [Online] \crlf Available: http://getspringy.com/
\stopitemize

\stoptext
